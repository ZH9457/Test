\documentclass[12pt, a4paper, oneside]{ctexart}
% \usepackage[margin=1in]{geometry}%设置边距,符合Word设定
\usepackage{geometry}
\geometry{left=2.54cm, right=2.54cm, top=3.18cm, bottom=3.18cm}
% \usepackage{ctex}
% \usepackage{setspace}
% \usepackage{lipsum}
\usepackage{graphicx}%插入图片
\usepackage{amsmath}%数学公式
\usepackage{amssymb}%数学公式
\usepackage{mathrsfs}%数学公式
\usepackage{bm}%数学公式

\title{A Simple Text}
\author{Your Name}
\date{\today}

\begin{document}

\maketitle

\section{\heiti\zihao{2} 前备知识}
\subsection*{\heiti\zihao{3} OFDM波形}
\songti OFDM是一种正交多载波调制技术,它将数据流进行串并变换,把高速数据流转换为多个低速数据流,
再将他们分别调制到多个正交子载波上进行传输。与传统的频率复用不同,OFDM不需要设置保护频带间隔,
甚至在频带上有所重叠,所以频谱利用率高,而且因为它是将信道划分为了多个正交子信道,每个信道的所占
带宽比较小,所以它能够很好地对抗频率选择性衰落和窄带干扰,此外,OFDM可以使用快速傅里叶变换进行调制
与解调,这能够很大程度上降低系统实现的复杂度。

因为频率选择性衰落会导致OFDM的少数子载波受到较大影响产生比特错误,所以一般OFDM还会采用在发送端进行
加扰的方式分散由衰落所带来的比特错误,让它在时间上类似均匀分布,这样可以提高系统的性能。而且在特定的子载波
上,通常会添加导频信号,接收端可以通过这些导频信号来进行信道估计,后续再做信道均衡可以大幅降低误码率。

由于多径时延,OFDM会产生符号间干扰,所以需要在时域相邻的OFDM符号间插入保护间隔来消除这部分干扰,一般
需要设置保护间隔的长度大于信道的最大多径时延,这样才能避免上一个符号进入到下一个符号的积分区域内,从而
消除符号间干扰。但若是直接让保护间隔保持空白,这样还会有子载波间干扰的问题,即一个OFDM符号在其时域积分区间
内各个子载波的正交性会受到破坏,所以通常会在保护间隔内插入循环前缀或者循环后缀来保证OFDM符号各子载波在积分区间内
的正交性,从而消除子载波间干扰。此外,添加循环前缀还能够使得OFDM符号的一部分产生周期性,这样可以将时域信号与信道
冲激响应的线性卷积转换为循环卷积,相当于在频域直接相乘,有利于后续直接做频域均衡。不过在实际工程中是很难得到信道的最大多径时延的,
所以一般都将保护间隔的时长设为信道的时延拓展均方根的2到4倍,其次,插入保护间隔的时间与OFDM符号时长的比率越大,
OFDM符号的有效性就越低,OFDM符号的时长也不能很长,因为时长越长,子载波间隔越小,一定带宽里包含的子载波越多,对硬件的要求也就越高,
同时OFDM符号对频率偏移越敏感,所以一般会设置保护间隔的时长为OFDM符号时长的1/5倍。

设OFDM的符号时长为$T$,子载波个数为$N$,则子载波间隔为$\Delta f = \dfrac{1}{T} $,单个OFDM符号可以表示如下:
\[
    s(t) = \sum_{n=0}^{N} d(n)\times e^{j2\pi n\Delta ft},t\in [0,T]
\]

\subsection*{\heiti\zihao{3} 信道估计}
\songti 信道估计是指通过对接收信号的处理,将信道对发送信号的影响用数学的方式表达出来。一般信道估计是
为了最小化某个误差值,与此同时尽量降低算法实现的复杂度。信道估计包括非盲估计、半盲估计和盲估计。非盲估计是指发送端发送
特定的训练序列,接收端通过接收信号与原始训练序列来进行信道估计,例如基于导频的信道估计就是接收端通过接受导频信号与原始
导频信号得到导频处的信道特性,再通过插值的方式得到数据子载波处或者其他发送符号时刻的信道特性。不过这种方法会占用数据传输
资源,导致传输效率降低。盲估计是指通过调制信号一些固有的、与传输数据无关的特性来进行信道估计,主要有基于子空间的和基于
最大期望的信道估计。这种方法需要接受端接收大量数据才能够统计特性来进行信道估计,且算法复杂度较高,不过因为它不占用传输资源,
所以相应的频谱利用率和数据有效性较高。半盲估计则是二者的折中,一般是使用较短的训练序列配合上盲估计算法来进行信道估计。

文章的这一部分主要聚焦于非盲信道估计,现在主要有三种方法,即最小二乘法信道估计算法(LS)、最小均方误差信道估计算法(MMSE)以及线性最小均方误差算法(LMMSE)。
首先介绍LS算法,假设接收信号为$Y=HX+Z$,信道估计为$\widehat{H}$,估计接收信号为$\widehat{Y}=\widehat{H}X$,误差函数可以表示如下:
$$
\begin{aligned}
    J(\widehat{H}) &=\left\| Y-\widehat{H}X \right\|^2 \\
    &=Y^HY-Y^HX\widehat{H}-\widehat{H}^HX^HY+\widehat{H}^HX^HX\widehat{H}
\end{aligned}
$$

而LS算法就是要使得这个误差函数最小,对上述误差函数求导可得:
$$
\dfrac{\partial J(\widehat{H})}{\partial\widehat{H}}=-2Y^HX+2\widehat{H}^HX^HX
$$

令其等于0,可得:
$$
\widehat{H}=(X^HX)^{-1}X^HY
$$

当X为满秩矩阵时,$\widehat{H}=X^{-1}Y$即为LS算法信道估计的结果。LS算法的均方误差(MSE)为:
$$
\begin{aligned}
    MSE_{LS} &=E\{(H-\widehat{H})^H(H-\widehat{H})\} \\
    &=E\{(X^{-1}Z)^H(X^{-1}Z)\} \\
    &=\frac{\sigma_Z^2}{\sigma_X^2} \\
    &=\frac{1}{SNR}
\end{aligned}
$$

由此可见,LS算法因为忽略了噪声的影响,所以计算复杂度较低,实现也较为简单,但这会导致估计误差受信噪比影响较大,特别是当信号衰落
较大、SNR较小时,LS算法的信道估计误差会很大,对信号的接受产生较大的影响。而MMSE算法则是考虑了噪声的影响,使用了一个加权矩阵调整
LS算法的估计结果,即$\widehat{H_M}=W\widehat{H}$,在MMSE算法中所要优化的目标函数为:
$$
J(\widehat{W})=\left\| H-W\widehat{H} \right\|^2
$$

当误差向量$e=H-W\widehat{H}$与$W\widehat{H}$正交时,上述目标函数可以取得最小值,所以问题转为求$E\{e\widehat{H}\}$等于零,即:
$$
\begin{aligned}
    E\{eW\widehat{H}\} &=E\{(H-W\widehat{H})\widehat{H}^H\} \\
    &=E\{H\widehat{H}^H\}-WE\{\widehat{H}\widehat{H}^H\} \\
    &=R_{H\widehat{H}}-WR_{\widehat{H}\widehat{H}} &=0
\end{aligned}
$$

也就是$W=R_{H\widehat{H}}R_{\widehat{H}\widehat{H}}^{-1}$,其中$R_{H\widehat{H}}$是$H$和$\widehat{H}$的互相关矩阵,$R_{\widehat{H}\widehat{H}}$是$\widehat{H}$的自相关矩阵。但是在使用MMSE算法进行信道估计时,需要不断地计算信噪比以及求矩阵的逆,
所以LMMSE就是在MMSE的基础上用统计信噪比的期望值代替实时信噪比,从而降低计算复杂度。

\section{\heiti\zihao{2} 系统模型}
\subsection*{\heiti\zihao{3} 模型概述}
\songti 对于感知而言,一个目标的速度通常是比较小的,所以对于最大不模糊速度的要求也不是那么高,而一个目标与感知站点相距较远的情况却时常发生,所以一般对于感知最大不模糊距离会有较高的要求,本文提出一种基于导频的通感一体的OFDM波形帧结构设计,在帧结构中使用步进导频来增大感知的不模糊距离,对于有时在少数子载波处会产生较大突变性衰落的信道,这种步进导频也能提高信道估计与信道均衡的性能,
从而提高通信性能,即使面对加性高斯白噪声信道,这种步进导频分布设计也不会对通信性能有明显影响,与此同时,文章还探究了使用步进导频导致出现虚假目标的原因并提出方法消除虚假目标。总的系统模型如图所示,包括通信与感知两个部分,二者在信号的发送端共用信号波形。发送端的流程包括生成随机二进制比特流、串并变换、星座映射、插入导频、IFFT变换、添加循环前缀。
对于通信方面,文章假设经过的是带有一定衰减和相位偏转的加性高斯白噪声信道,在接收端的处理流程包括去除循环前缀、FFT变换、信道估计、信道均衡、反星座映射、并串变换。对于感知方面,信道考虑了时延、多普勒频移、散射系数以及加性高斯白噪声,接收端的处理流程包括去除循环前缀、FFT变换、抽取导频、压缩步进导频、
2D-FFT处理、消除虚假目标。
\subsection*{\heiti\zihao{3} 发送信号}
\songti 设OFDM的符号时长为$T$,子载波个数为$N$,一共有$M$个OFDM符号,循环前缀的长度为$Q$,每个OFDM符号中的采样间隔为$T_S=\frac{T}{N}$,则子载波间隔为$\Delta f = \dfrac{1}{T} $,那么单个离散OFDM符号可以表示如下:
\[
    s(m) = \sum_{n=0}^{N} d(n)\times e^{j2\pi \frac{nm}{N}},m=0,1,\cdots,N-1
\]

其中$d(n)$为OFDM符号的调制数据,包括了导频数据和用户数据,插入和抽取导频需要知道导频的位置,现在假设导频间隔为$P$,且子载波数$N$能被导频间隔整除,导频起始索引为0,那么导频的位置索引可以表示如:
\[
    n_i = i\times P,i=0,1,\cdots,\frac{N}{P}-1
\]

现将步进数$Step$设定为导频插入位置的循环周期,例如$Step=4$时,代表每4个OFDM符号为一个导频位置周期。但需要注意,导频间隔必须能被步进数整除,当$Step=\beta$时,导频位置索引可以表示如:
\begin{align}
    n_{i,m}&=j\times P'+i\times P,i=0,1,\cdots,\frac{N}{P}-1 \\
    j&=(m)mod(\beta),m=0,1,\cdots,M-1 \\
    P'&=\frac{P}{Step}
\end{align}

其中$(x)mod(y)$表示$x$对$y$取模,$m$表示第$m$个OFDM符号。在时频资源中插好导频信号之后,在其余位置上插入用户数据,总的发送信号可以表示如:
$$
X'=\begin{bmatrix}
    C_{Q\times M} \\
    X_{N\times M}
\end{bmatrix},X=\begin{bmatrix}
    S \\
    C
\end{bmatrix}_{N\times M}
$$

其中,$C$表示循环前缀矩阵,$X$表示刚经过IFFT变换得到的时域信号矩阵,$X'$表示添加循环前缀之后的待发送信号矩阵。将信号矩阵转换为信号向量:
$$
x=vec(X')
$$

其中,$vec(.)$表示将矩阵转换为向量的函数。$x$即为发送端的发送信号。
\subsection*{\heiti\zihao{3} 信道}
信号经过通信加性高斯白噪声信道得到的接收信号为:
$$
y_c=Ax+z
$$

其中,$A$表示信道衰减系数,$z$表示高斯白噪声,其均值为0,方差为$\sigma^2$。

对于感知,我们先考虑单个目标的情况,设总的信号采样个数为$K=(N+Q)\times M$,目标的时延为$L$个采样单位,多普勒频率为$f_d$,信号经过感知信道后得到的接收信号为:
$$
\begin{align}
y_s&=\alpha Dx\odot d+z \\
D&=\begin{bmatrix}
    0_{L\times K} \\
    I_{(K-L)\times K}
\end{bmatrix}
\end{align}
$$

\subsection*{\heiti\zihao{3} 接收端处理}
在接收端,先将接收信号重构为$(N+Q)\times M$的矩阵,去除矩阵的前$Q$行,即去除循环前缀,然后对矩阵进行列向的FFT变换,处理之后得到的结果可以表示为$Y_{N\times M}$。
\subsubsection*{\heiti\zihao{4} 通信接收端处理}
在通信方面,在前面处理得到的结果中抽取导频信号,抽取导频信号的索引如公式所示,后续在进行信道估计的时候,我们采用LS算法,并使用一次线性插值算法进行频域插值,得到信道估计矩阵$\widehat{H}$,
然后使用迫零算法进行信道均衡,补偿信道对信号带来的损失,最后通过反星座映射和并串变换得到接受比特信号。
\subsubsection*{\heiti\zihao{4} 感知接收端处理}
首先,对于一个离散信号$x(n)$,有DFT变换如下:
$$
DFT\{x(n)\}=\sum_{n=0}^{N-1}x(n)e^{-j2\pi \frac{nm}{N}}
$$

当其在时域发生循环移位$L$位时,有:在频域上表现为每一项都多了一个线性相移变量,即:
$$
\begin{aligned}
DFT\{x(n-L)\}&=\sum_{n=0}^{N-1}x(n-L)e^{-j2\pi \frac{nm}{N}} \\
&=e^{-j2\pi \frac{Lm}{N}}\sum_{n=0}^{N-1}x(n-L)e^{-j2\pi \frac{(n-L)m}{N}} \\
\end{aligned}
$$
\end{document}